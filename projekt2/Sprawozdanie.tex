% Preamble
\documentclass[11pt]{article}

% Packages
\usepackage{amsmath}
\usepackage[a4paper, margin=0.5in]{geometry}
\usepackage{graphicx}
\usepackage[utf8]{inputenc}
\usepackage[T1]{fontenc}
\usepackage[polish]{babel}
\usepackage{float}
\usepackage{hyperref}
\usepackage{cleveref}

\title{Projekt 2. Promethee/Electre}
\author{Oskar Kiliańczyk 151863 \& Wojciech Kot 151876}
\date{}

% Document
\begin{document}

\maketitle
\newpage

\section{Zbiór danych}
Zbiór danych
\subsection{Jaka jest domena problemu?}

\subsection{Jakie jest źródło danych?}

\subsection{Jaki jest punkt widzenia decydenta?}

\subsection{Ile wariantów decyzyjnych zostało uwzględnionych? Czy w oryginalnym zbiorze danych było ich więcej?}

\subsection{Opisz jeden z rozważanych wariantów decyzyjnych (podaj jego nazwę, oceny oraz określ preferencje dla tego wariantu).}

\subsection{Ile kryteriów zostało uwzględnionych? Czy w oryginalnym zbiorze danych było ich więcej?}

\subsection{Jakie są dziedziny poszczególnych kryteriów (dyskretne / ciągłe)? Uwaga: w przypadku dziedzin ciągłych określ zakres zmienności kryterium, w innych przypadkach podaj możliwe wartości. Jaki jest charakter poszczególnych kryteriów (zysk / koszt)?}

\subsection{Czy wszystkie kryteria mają jednakowe znaczenie (czy powinny mieć takie same „wagi”)? Jeśli nie, czy względne znaczenie kryteriów można wyrazić za pomocą wag? W takim przypadku oszacuj wagi każdego kryterium w skali od 1 do 10. Czy wśród kryteriów są takie, które są całkowicie lub prawie nieistotne?}

\subsection{Czy w rozważanym zbiorze danych występują zdominowane warianty decyzyjne? Jeśli tak, przedstaw wszystkie (warianty dominujące i zdominowane), podając ich nazwy oraz wartości dla poszczególnych kryteriów.}

\subsection{Jak według Ciebie powinien wyglądać teoretycznie najlepszy wariant decyzyjny? Czy powinien mieć niewielką przewagę w wielu kryteriach, czy raczej silną przewagę w kilku (ale kluczowych) kryteriach? Których?}

\subsection{Który z rozważanych wariantów decyzyjnych (podaj nazwę i wartości dla poszczególnych kryteriów) wydaje się najlepszy / zdecydowanie lepszy od pozostałych? Czy decyduje o tym jeden czynnik (np. zdecydowanie najniższa cena), czy raczej ogólna wartość kryteriów? Czy ten wariant ma jakieś słabe strony?}

\subsection{Który z rozważanych wariantów decyzyjnych (podaj nazwę i wartości dla poszczególnych kryteriów) wydaje się najgorszy / zdecydowanie gorszy od pozostałych? Czy decyduje o tym jeden czynnik (np. zdecydowanie najwyższa cena), czy raczej ogólna wartość kryteriów? Czy ten wariant ma jakieś mocne strony?}

\subsection{Podaj co najmniej 4 porównania parami pomiędzy wariantami w Twoim zbiorze danych.}

\section{Promethee}
Promethee
\subsection{Opisz informacje preferencyjną podaną na wejściu metod}

\subsection{Podaj ostateczny wynik uzyskany za pomocą metod}

\subsection{Porównaj ranking cząstkowy i zupełny (Promethee I i Promethee II)}

\subsection{Porównaj wyniki uzyskane za pomocą zaimplementowanej metody z wariantami, które zidentyfikowałeś jako najlepszą i najgorszą (w sekcji Zbiór danych)}

\subsection{Porównaj wyniki uzyskane za pomocą zaimplementowanej metody z wcześniejszymi przekonaniami na temat wariantów (porównania parami określone w sekcji Zbiór danych)}

\subsection{Dodatkowe uwagi na temat uzyskanych wyników. Możesz odnieść się m.in. do wariantów, które cię zaskoczyły, lub do pozycji w rankingu, które uważasz za niepoprawne.}

\section{Electre III}
Electre III
\subsection{Opisz informacje preferencyjne podane na wejściu metody}

\subsection{Podaj ostateczny wynik uzyskany za pomocą metody}

\subsection{Skomentuj ostateczne wyniki metody}

\subsection{Porównaj wyniki uzyskane za pomocą zaimplementowanej metody z wariantami, które zidentyfikowałeś jako najlepszy i najgorszy (w sekcji Zbiór danych)}

\subsection{Porównaj wyniki uzyskane za pomocą zaimplementowanej metody z wcześniejszymi przekonaniami na temat wariantów (porównania parami określone w sekcji Zbiór danych)}

\subsection{Dodatkowe uwagi na temat uzyskanych wyników. Możesz odnieść się m.in. do wariantów, które cię zaskoczyły, lub do pozycji w rankingu, które uważasz za niepoprawne.}

\section{Porównanie}
Porównanie
\subsection{Zgodność między metodami}

\subsection{Różnice między metodami}

\subsection{Dodatkowe uwagi na temat wyników}


\section{Link do repozytorium}\label{sec:link-do-repo}
Kod źródłowy w repozytorium GitHub dostępny pod linkiem: \\
\href{https://github.com/KotZPolibudy/PUT_ISWD/tree/main/projekt2}{Repozytorium ISWD - projekt 2}.


\end{document}
