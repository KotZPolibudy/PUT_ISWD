\documentclass[a4paper,12pt]{article}

\usepackage{graphicx} 
\usepackage[T1]{fontenc}
\usepackage{polski}
\usepackage[utf8]{inputenc}
\usepackage[a4paper]{geometry}
\usepackage{xcolor}
\usepackage{float}

\usepackage{graphicx} 
\usepackage[T1]{fontenc}
\usepackage[polish]{babel}
\usepackage[utf8]{inputenc}
\usepackage{caption}
\usepackage[a4paper]{geometry}
\usepackage{amsmath}
\usepackage[section]{placeins}
\usepackage[tagged, highstructure]{accessibility}
\usepackage{tabularx}
\usepackage{pdfcomment}
\usepackage{xcolor}
\usepackage{float}

\author{\textcolor{blue}{Oskar Kiliańczyk 151863 \& Wojciech Kot 151879}}
\date{}

\title{Graniczna Analiza Danych - raport}

\begin{document}
\maketitle

\section{Efektywność}
W poniższej tabeli przedstaw otrzymane wartości efektywności dla wszystkich lotnisk.
\begin{table}[H]
    \centering
    \begin{tabular}{c|c}
    \hline
         Lotnisko & Efektywność  \\ \hline
         WAW & 1.000 \\
        KRK & 1.000 \\
        KAT & 0.591 \\
        WRO & 1.000 \\
        POZ & 0.800 \\
        LCJ & 0.300 \\
        GDN & 1.000 \\
        SZZ & 0.271 \\
        BZG & 1.000 \\
        RZE & 0.409 \\
        IEG & 0.258 \\
         \hline
    \end{tabular}
    \caption{Wartości efektywności dla analizowanych lotnisk}
    \label{tab:airports-efficiency}
\end{table}

Lotniska efektywne: WAW, KRK, WRO, GDN, BZG \\
Lotniska nieefektywne: KAT, POZ, LCJ, SZZ, RZE, IEG

\section{Hipotetyczna jednostka porównawcza oraz potrzebne poprawki}

\begin{table}[H]
    \centering
    \begin{tabular}{c|cccc|cccc}
    \hline
        & \multicolumn{4}{c|}{HCU} & \multicolumn{4}{c}{Poprawki}  \\
         Lotnisko & $i_1$ & $i_2$ & $i_3$ & $i_4$ & $i_1$ & $i_2$ & $i_3$ & $i_4$ \\ \hline
         KAT & 2.13 & 18.92 & 33.94 & 4.40 & 1.47 & 13.08 & 23.46 & 6.10 \\
POZ & 1.20 & 8.00 & 19.20 & 1.93 & 0.30 & 2.00 & 4.80 & 2.07 \\
LCJ & 0.18 & 2.78 & 7.20 & 0.47 & 0.42 & 9.22 & 16.80 & 3.43 \\
SZZ & 0.19 & 2.71 & 6.96 & 0.47 & 0.51 & 7.29 & 18.74 & 1.43 \\
RZE & 0.25 & 2.46 & 4.62 & 0.54 & 0.35 & 3.54 & 6.68 & 2.16 \\
IEG & 0.03 & 0.39 & 1.11 & 0.06 & 0.07 & 9.61 & 62.29 & 2.94 \\
    \end{tabular}
    \caption{Wartości wejść hipotetycznej jednostki porównawczej oraz poprawki potrzebne do osiągnięcia efektywności dla nieefektywnych lotnisk }
    \label{tab:airports-hcu-and-improvements}
\end{table}

\section{Superefektywność}

Otrzymane wartości superefektywności dla lotnisk w tabeli poniżej.

\textcolor{blue}{TODO - TO JEST ABSOLUTNIE ŹLE POLICZONE}

\begin{table}[H]
    \centering
    \begin{tabular}{c|c}
    \hline
         Lotnisko & Superefektywność  \\ \hline
         WAW & ...\\
         \hline
    \end{tabular}
    \caption{Wartości superefektywności dla analizowanych lotnisk}
    \label{tab:airports-super-efficiency}
\end{table}

\section{Efektywność krzyżowa}

\textcolor{blue}{W poniższej tabeli przedstaw macierz efektywności krzyżowych dla wszystkich lotnisk oraz ich średnie efektywności krzyżowa.}    

\begin{table}[H]
\resizebox{\textwidth}{!}{
\begin{tabular}{c|ccccccccccc|c}
\hline
& WAW & KRK & KAT & WRO & POZ & LCJ & GDN &SZZ & BZG & RZE & IEG & $CR_{avg}$ \\ \hline
WAW & 1.000 & \ldots & \ldots & \ldots & \ldots & \ldots & \ldots & \ldots & \ldots & \ldots & \ldots & \ldots \\
\ldots & \ldots & \ldots & \ldots & \ldots & \ldots & \ldots & \ldots & \ldots & \ldots & \ldots & \ldots & \ldots \\ 
\hline
\end{tabular}}
\label{tab:airports-cross-efficiency}
\end{table}
            
\section{Rozkład efektywności}
\textcolor{blue}{W tej sekcji pokaż wyniki rozkładu efektywności (podział na 5 przedziałów) oraz oszacowaną oczekiwaną wartość efektywności dla wszystkich lotnisk.}
\begin{table}[H]
\begin{tabular}{c|ccccc|c}
\hline
    & $[0-0.2)$ & $[0.2-0.4)$ & $[0.4-0.6)$ & $[0.6-0.8)$ & $[0.8-1.0]$ & $EE$    \\ \hline
WAW & 0.00      & \ldots & \ldots & \ldots & \ldots & \ldots \\
\ldots & \ldots & \ldots & \ldots & \ldots & \ldots & \ldots \\
\hline
\end{tabular}
\label{tab:efficiency-distribution}
\end{table}

\section{Rankingi jednostek}
\textcolor{blue}{Przedstaw i porównaj rankingi lotnisk uzyskane różnymi metodami (superefektywność, średnia efektywność krzyżowa oraz oczekiwana wartość efektywności).}

\noindent Superefektywność: $WAW \succ \ldots \succ \ldots$ \\
Średnia efektywność krzyżowa: \ldots \\
Oczekiwana wartość efektywności: \ldots \\

\textcolor{blue}{Tu przedstaw wnioski z porównania rankingów.}

\end{document}