\documentclass[a4paper,12pt]{article}

\usepackage{graphicx} 
\usepackage[T1]{fontenc}
\usepackage{polski}
\usepackage[utf8]{inputenc}
\usepackage[a4paper]{geometry}
\usepackage{xcolor}
\usepackage{float}

\usepackage{graphicx} 
\usepackage[T1]{fontenc}
\usepackage[polish]{babel}
\usepackage[utf8]{inputenc}
\usepackage{caption}
\usepackage[a4paper]{geometry}
\usepackage{amsmath}
\usepackage[section]{placeins}
\usepackage[tagged, highstructure]{accessibility}
\usepackage{tabularx}
\usepackage{pdfcomment}
\usepackage{xcolor}
\usepackage{float}

\author{\textcolor{blue}{Oskar Kiliańczyk 151863 \& Wojciech Kot 151879}}
\date{}

\title{Graniczna Analiza Danych - raport}

\begin{document}
\maketitle

\section{Efektywność}
W poniższej tabeli przedstaw otrzymane wartości efektywności dla wszystkich lotnisk.
\begin{table}[H]
    \centering
    \begin{tabular}{c|c}
    \hline
         Lotnisko & Efektywność  \\ \hline
         WAW & 1.000 \\
        KRK & 1.000 \\
        KAT & 0.591 \\
        WRO & 1.000 \\
        POZ & 0.800 \\
        LCJ & 0.300 \\
        GDN & 1.000 \\
        SZZ & 0.271 \\
        BZG & 1.000 \\
        RZE & 0.409 \\
        IEG & 0.258 \\
         \hline
    \end{tabular}
    \caption{Wartości efektywności dla analizowanych lotnisk}
    \label{tab:airports-efficiency}
\end{table}

Lotniska efektywne: WAW, KRK, WRO, GDN, BZG \\
Lotniska nieefektywne: KAT, POZ, LCJ, SZZ, RZE, IEG

\section{Hipotetyczna jednostka porównawcza oraz potrzebne poprawki}

\begin{table}[H]
    \centering
    \begin{tabular}{c|cccc|cccc}
    \hline
        & \multicolumn{4}{c|}{HCU} & \multicolumn{4}{c}{Poprawki}  \\
         Lotnisko & $i_1$ & $i_2$ & $i_3$ & $i_4$ & $i_1$ & $i_2$ & $i_3$ & $i_4$ \\ \hline
         KAT & 2.13 & 18.92 & 33.94 & 4.40 & 1.47 & 13.08 & 23.46 & 6.10 \\
POZ & 1.20 & 8.00 & 19.20 & 1.93 & 0.30 & 2.00 & 4.80 & 2.07 \\
LCJ & 0.18 & 2.78 & 7.20 & 0.47 & 0.42 & 9.22 & 16.80 & 3.43 \\
SZZ & 0.19 & 2.71 & 6.96 & 0.47 & 0.51 & 7.29 & 18.74 & 1.43 \\
RZE & 0.25 & 2.46 & 4.62 & 0.54 & 0.35 & 3.54 & 6.68 & 2.16 \\
IEG & 0.03 & 0.39 & 1.11 & 0.06 & 0.07 & 9.61 & 62.29 & 2.94 \\
    \end{tabular}
    \caption{Wartości wejść hipotetycznej jednostki porównawczej oraz poprawki potrzebne do osiągnięcia efektywności dla nieefektywnych lotnisk }
    \label{tab:airports-hcu-and-improvements}
\end{table}

\section{Superefektywność}

Otrzymane wartości superefektywności dla lotnisk w tabeli poniżej.

\begin{table}[H]
    \centering
    \begin{tabular}{c|c}
    \hline
         Lotnisko & Superefektywność  \\ \hline
         WAW & 2.278 \\
KRK & 1.124 \\
KAT & 0.591 \\
WRO & 1.040 \\
POZ & 0.800 \\
LCJ & 0.300 \\
GDN & 2.000 \\
SZZ & 0.271 \\
BZG & 1.746 \\
RZE & 0.409 \\
IEG & 0.258 \\
         \hline
    \end{tabular}
    \caption{Wartości superefektywności dla analizowanych lotnisk}
    \label{tab:airports-super-efficiency}
\end{table}

\section{Efektywność krzyżowa}

W poniższej tabeli macierz efektywności krzyżowych dla wszystkich lotnisk oraz ich średnie efektywności krzyżowa.

\begin{table}[H]
\resizebox{\textwidth}{!}{
\begin{tabular}{c|ccccccccccc|c}
\hline
& WAW & KRK & KAT & WRO & POZ & LCJ & GDN &SZZ & BZG & RZE & IEG & $CR_{avg}$ \\ \hline
WAW & 1.000 & 1.000 & 0.913 & 1.000 & 1.000 & 0.595 & 0.452 & 1.000 & 0.595 & 0.903 & 0.523 & 0.816 \\
KRK & 0.806 & 1.000 & 1.000 & 1.000 & 1.000 & 0.491 & 0.468 & 0.755 & 0.491 & 0.996 & 0.428 & 0.767 \\
KAT & 0.469 & 0.575 & 0.591 & 0.563 & 0.563 & 0.278 & 0.333 & 0.371 & 0.278 & 0.591 & 0.248 & 0.442 \\
WRO & 0.748 & 0.965 & 1.000 & 1.000 & 1.000 & 0.605 & 0.500 & 0.856 & 0.605 & 1.000 & 0.531 & 0.801 \\
POZ & 0.716 & 0.793 & 0.774 & 0.800 & 0.800 & 0.512 & 0.433 & 0.737 & 0.512 & 0.770 & 0.458 & 0.664 \\
LCJ & 0.202 & 0.240 & 0.259 & 0.255 & 0.255 & 0.300 & 0.250 & 0.273 & 0.300 & 0.261 & 0.297 & 0.263 \\
GDN & 1.000 & 1.000 & 1.000 & 1.000 & 1.000 & 1.000 & 1.000 & 1.000 & 1.000 & 1.000 & 1.000 & 1.000 \\
SZZ & 0.222 & 0.234 & 0.238 & 0.243 & 0.243 & 0.261 & 0.214 & 0.271 & 0.261 & 0.238 & 0.254 & 0.244 \\
BZG & 0.404 & 0.721 & 0.972 & 0.909 & 0.909 & 1.000 & 0.500 & 1.000 & 1.000 & 1.000 & 0.876 & 0.845 \\
RZE & 0.327 & 0.396 & 0.409 & 0.403 & 0.403 & 0.273 & 0.250 & 0.346 & 0.273 & 0.409 & 0.247 & 0.340 \\
IEG & 0.005 & 0.003 & 0.002 & 0.006 & 0.006 & 0.078 & 0.025 & 0.036 & 0.078 & 0.002 & 0.258 & 0.045 \\
\hline
\end{tabular}}
\label{tab:airports-cross-efficiency}
\end{table}
            
\section{Rozkład efektywności}
W tej sekcji wyniki rozkładu efektywności (podział na 5 przedziałów) oraz oszacowana oczekiwana wartość efektywności dla wszystkich lotnisk.
\begin{table}[H]
\begin{tabular}{c|ccccc|c}
\hline
    & $[0-0.2)$ & $[0.2-0.4)$ & $[0.4-0.6)$ & $[0.6-0.8)$ & $[0.8-1.0]$ & $EE$    \\ \hline
WAW & 0.00 & 0.00 & 0.36 & 0.00 & 0.64 & 0.80 \\
KRK & 0.00 & 0.00 & 0.36 & 0.18 & 0.45 & 0.75 \\
KAT & 0.00 & 0.45 & 0.55 & 0.00 & 0.00 & 0.43 \\
WRO & 0.00 & 0.00 & 0.18 & 0.27 & 0.55 & 0.78 \\
POZ & 0.00 & 0.00 & 0.36 & 0.64 & 0.00 & 0.65 \\
LCJ & 0.00 & 1.00 & 0.00 & 0.00 & 0.00 & 0.26 \\
GDN & 0.00 & 0.00 & 0.00 & 0.00 & 1.00 & 1.00 \\
SZZ & 0.00 & 1.00 & 0.00 & 0.00 & 0.00 & 0.24 \\
BZG & 0.00 & 0.00 & 0.18 & 0.09 & 0.73 & 0.83 \\
RZE & 0.00 & 0.73 & 0.27 & 0.00 & 0.00 & 0.33 \\
IEG & 1.00 & 0.00 & 0.00 & 0.00 & 0.00 & 0.03 \\
\hline
\end{tabular}
\label{tab:efficiency-distribution}
\end{table}

\section{Rankingi jednostek}
Rankingi lotnisk uzyskane różnymi metodami (superefektywność, średnia efektywność krzyżowa oraz oczekiwana wartość efektywności).

\noindent Superefektywność:\\
$WAW \succ GDN \succ BZG \succ KRK \succ WRO \succ POZ \succ KAT \succ RZE \succ LCJ \succ SZZ \succ IEG$ \\
Średnia efektywność krzyżowa:\\
$GDN \succ BZG \succ WAW \succ WRO \succ KRK \succ POZ \succ KAT \succ RZE \succ LCJ \succ SZZ \succ IEG$ \\
Oczekiwana wartość efektywności:\\
$WAW \succ KRK \succ WRO \succ BZG \succ GDN \succ POZ \succ KAT \succ RZE \succ LCJ \succ SZZ \succ IEG$ \\

\section{Wnioski z porównania rankingów}

    Przede wszystkim rzucające się w oczy jest to, że każdy z rankingów kończy się tą samą sekwencją:\\
    $POZ \succ KAT \succ RZE \succ LCJ \succ SZZ \succ IEG$ \\
    co sugeruje, że IEG (Zielona Góra) rzeczywiście będzie najmniej efektywnym lotniskiem, a wszystkie z tych lotnisk mają coś od poprawy, szczególnie te bliżej końca.\\

    Dodatkowo, WAW (Warszawa) osiąga pierwsze miejsca w dwóch rankingach, i jedno trzecie, co sugeruje że jest liderem.\\
    GDN plasuje się na pierwszym, drugim i raz na piątym miejscu, co również sugeruje jego silną pozycję.\\

    Poza dwoma wyżej wymienionymi lotniskami, wysokie miejsca zajmują również BZG, KRK oraz WRO, co również sugeruje ich silną pozycję wśród lotnisk, ale nie tak silną jak wcześniej wymienionych.


\end{document}