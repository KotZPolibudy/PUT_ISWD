% Preamble
\documentclass[11pt]{article}

% Packages
\usepackage{amsmath}
\usepackage[a4paper, margin=0.5in]{geometry}
\usepackage{graphicx} % daj an 1in jak chcesz normalniejszy margines, ale kod mi się w linii nie mieści :P
\usepackage[utf8]{inputenc}
\usepackage[T1]{fontenc}
\usepackage[polish]{babel}
\usepackage{float}
\usepackage{hyperref}
\usepackage{cleveref}

\title{Projekt 1. UTA}
\author{Oskar Kiliańczyk 151863 \& Wojciech Kot 151876}
\date{}

% Document
\begin{document}

\maketitle
\newpage

\section{Opis informacji preferencyjnej}\label{sec:opis-informacji-preferencyjnej}

Z racji że obaj wylosowaliśmy informację preferencyjną nr.\ 4, to firma przede wszystkim skupia się na preferowanej lokacji, gdzie:
Lokalizacja R2 jest preferowana nad R1 oraz R1 nad R3.
Jako drugie, dodatkowe kryterium przyjęliśmy sposób finansowania,
uznając metodę F1 (kWh-fee method) jako preferowaną ponad zarówno F2 (prorata method) oraz F3 (waste-fee method).


Pary referencyjne dobrane do naszego problemu to pary:
    przydzielone nam odgórnie:
    11 i 14, gdzie 14 jest preferowane ponad 11 (R2 > R1)
    2 i 25, gdzie 2 jest preferowane ponad 25 (R1 > R3)
    oraz wybrane przez nas:
    11 i 17, gdzie 11 jest preferowane nad 17 (R2 > R3)
    oraz dwie dla drugiego w ważności kryterium:
    4 i 5, gdzie 4 jest preferowane nad 5 (R2=R2, F1 > F2)
    4 i 6,  gdzie 4 jest preferowane nad 6 (R2=R2, F1 > F3)

Na wartości wag dodaliśmy dodatkowe ograniczenia w postaci:
- wymuszenia monotoniczności (wszystkie kryteria są typu koszt).
- normalizacji wag (aby użyteczność idealnego wariantu wynosiła 1, a antyidealnego 0).
- dla każdego kryterium waga idealnego wariantu nie może być większa niż 0.5 (zapewniamy brak dominującego kryterium)
ani mniejsza niż 0.1 (zapewniamy że każde kryterium jest w jakimś stopniu ważne)


\section{Wynik uzyskany z solvera}\label{sec:wynik-uzyskany-z-solvera}

TODO wstawić wykresiki i podsumować wartości użyteczności
(w princie pewnie przyda się posortować po użyteczności, bo tego nie zrobiłem, więc słabo się to będzie czytać)


\section{Wyniki}\label{sec:wyniki}

Wnioski - wnioski

\section{Link do repozytorium}\label{sec:link-do-repo}
Kod źródłowy w repozytorium GitHub dostępny pod linkiem: \\
\href{https://github.com/KotZPolibudy/PUT_ISWD/tree/main/projekt1}{Repozytorium ISWD - projekt 1}.


\section{TODOS}\label{sec:todos}

**To wszystko trzeba zrobić, to jest wariant na 3**

• Wynik uzyskany z solvera:
– wartości użyteczności wszystkich wariantów zarówno referencyjnych jak i niereferencyjnych;
– ranking wszystkich wariantów;
– wartość funkcji celu;
– wykresy cząstkowych funkcji użyteczności.

• Krótkie podsumowanie wyników:
– sprawdzenie zgodności wyników z podaną informacją preferencyjną;
– sprawdzenie, czy otrzymany ranking jest spójny z informacją preferencyjną dla kilku wybranych wariantów niereferencyjnych.
– jaka strategia została oceniona jako najlepsza i najgorsza;
– jaki był wpływ kryteriów na ostateczny wynik.

W przypadku braku istnienia spójnego modelu dla zadanej informacji preferencyjnej należy taką
informację zamieścić w raporcie i zmienić informację preferencyjną

\end{document}
